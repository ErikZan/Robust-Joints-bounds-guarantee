\documentclass{article}
\usepackage[utf8]{inputenc}
\usepackage{import}
\usepackage{graphics} % for pdf, bitmapped graphics files
\usepackage{epsfig} % for postscript graphics files
\usepackage{mathptmx} % assumes new font selection scheme installed
\usepackage{times} % assumes new font selection scheme installed
\usepackage{amsmath} % assumes amsmath package installed
\usepackage{amssymb}  % assumes amsmath package installed
\usepackage{biblatex}
\addbibresource{Reference.bib}
% DEL PRETE PACKAGES
\usepackage{amssymb}  % assumes amsmath package installed
\usepackage{colortbl}	% to color table background
\usepackage[table]{xcolor}
\usepackage[noend]{algpseudocode}
\usepackage{algorithm}
\usepackage[caption=true]{subfig}
\usepackage[pdftex]{graphicx}

\graphicspath{ {figures/} }
\input{./math_commands.tex}

\title{Roboust joint bounds guarantee}
\author{Erik Zanolli}
\date{October 2020}

\begin{document}

\maketitle
\section{Introduction}
The paper deals with the problem of a robust controlling of robotic system with bounds on its joints position,velocity and acceleration and subject to a disturbances. The problem of controlling a robotic system constrained in position,velocity and acceleration has been already discussed in the literature. The main contribution of this paper is to extend the previous obtained results introducing a method to deal with disturbances on the input
\section{State of the art}
[recap on other paper]
The approach of our interest is the one developed in the \cite{DelPrete2018} because
\begin{itemize}
    \item It is exact and not introduce any type of arbitrary conservatism
    \item It assumes a constant acceleration between consecutive timestep, so it bounds the real acceleration
\end{itemize}
\section{Problem Statement}
\subsection{Disturbance definition}
Is fundamental, to understand and treat the correctly the problem, to define the disturbances that the manuscript is going to consider and deal with. Such disturbance that we are going to introduce are on the inputs of the system, so on accelerations given to the joints. Our initial hypothesis is that the disturbance, defined from now on as $w$, is bounded and its bounds are known: $w \in [-\bar{w},\bar{w}]$. This hypothesis is reasonable under the assumption that the disturbance is "small" and quite predictable i.e. we don't know the exact value but we are quite sure that it can contained in the $[-\bar{w},\bar{w}]$ interval. This allows us to deal initially with a much simpler problem. The disturbance $w$, such as the acceleration, remains constant thorough the whole timestep. Hereafter the future position and the future velocity assuming a constant acceleration and a constant disturbance thorough the timestep are shown:
\begin{equation} \label{eq:simple_law} \begin{aligned} 
    q_{i+1} &= q_i + \dt \, \dq_i + \half \dt^2 (\ddq_i+w) \\
    \dq_{i+1} &= \dq_i + \dt \, (\ddq_i+w)
    \end{aligned} 
\end{equation}%, where the $\bar{w}$ value represent the bound value of 
\subsection{Feasible States}
Considering a robot whose joints has limited position and limited accelerations its possible to define as in \cite{DelPrete2018} the set of feasible state for as \mathcal{F}
\begin{equation}
    \begin{aligned} 
        \mathcal{F} = \{ (q,\dq) \in \Rv{2} : q^{min} \le q \le q^{max}, |\dq | \le \dq^{max} \}
    \end{aligned}
\end{equation}
As previously stated our control inputs are accelerations on joints and they are also bounded: $|\ddq|\leq\ddq^{max}$. It is important to notice that $w$ does not influence this definition, indeed $\ddq$ is the acceleration applied by the joint and to which we add the disturbance. 
 As shown in \cite{DelPrete2018} this problem of finding the maximum and minimum allowable accelerations such that the constraint on position and velocity are satisfied in the future can be formulated as an infinite-horizon optimal control problem 
 \begin{equation} \label{eq:original_problem} \begin{aligned} 
     \ddq_0^{max} = \maximize_{\ddq_0, \ddq_1, \dots} \quad & \ddq_0 & \\
     \st \quad & (q(t), \dq(t)) \in \mathcal{F}			& \forall t > 0\\
     & |\ddq_i | \le \ddq^{max}  						& \forall i \ge 0\\
     & (q(0), \dq(0)) \quad \text{fixed}
     \end{aligned} 
 \end{equation}
As it is possible to see so far the definition of the feasible states and the initial formulation of the problem is still not affected by the presence of $w$. The contribute become evident looking at the $q$ time laws, such as in \ref{eq:simple_law} and the $q$ time law inside the timestep:
\begin{equation} \begin{aligned} 
    q(i \dt + t) &= q_i + t \, \dq_i + \frac{t^2}{2} (\ddq_i+w) 	\quad & \forall i \ge 0, t \in [0, \dt] \\
    \dq(i \dt + t) &= \dq_i + t \, (\ddq_i+w)     				\quad & \forall i \ge 0, t \in [0, \dt]
    \end{aligned} 
    \end{equation}
The $q$ behavior inside the timestep is fundamental, because the joints limits must be respect also in continuous time inside the timestep and not only in discrete time. This problem has an infinite number of constraints and cannot be solved, as stated in as discussed in \cite{DelPrete2018}. Our approach to the solution is to extend the result obtained by \cite{DelPrete2018} taking into account the disturbances, so recompute the viability kernel proposed and re adapt the position and velocities inequalities.
\section{Problem solution}
As stated in \cite{DelPrete2018} the concept of viability will help us reformulate the \ref{eq:original_problem}. A state is defined as viable if starting from that state there exists a sequence of control inputs that allows for the satisfaction of all constraints in the future. The viability kernel $\mathcal{V}$ is defined as
\begin{equation} \begin{aligned}
    (q(0),\dq(0)) \in \mathcal{V}     \quad \Leftrightarrow \quad \exists (\ddq_i)_{i=0}^\infty :  (q(t), \dq(t)) \in \mathcal{F} \quad &\forall t \ge 0, \\ 
    |\ddot{q}_i| \le \ddot{q}^{max} \quad &\forall i \ge 0
    \end{aligned} 
\end{equation}
The viability kernel ensure the existence of a feasible future trajectory, so ensure that the next state belongs to $\mathcal{V}$ and solve our initial problem. But this definition of $\mathcal{V}$ is of no practical utility.% 
We must derive an equivalent definition for this problem. In particular we can reformulate the problem as the simple problem of ensuring that the next state is viable.
\subsubsection*{Continuous time control}
In the beginning let us assume that we can change $\dq$ at any instant. This result in a set of viable state $\mathcal{V}^C$ that is a superset respect to the previous one, so $\mathcal{V} \subset \mathcal{V}^C$. It's obvious that a viable state is also feasible, but not all the feasible state are viable i.e. if the system is approaching a bound with a large velocity even the maximum deceleration capabilities of system may not be able to stop and prevent a violation of the constraint. For a given initial position $q_0$, we can find the maximum initial velocity $\dq^{\mathcal{V}}_M$ that allows us to satisfy the position limits in the future:
\begin{equation} \begin{aligned} 
\dq^{\mathcal{V}}_M = \maximize_{\dq_0, \ddq(t)} \quad & \dq_0 & \\
\st \quad 
& (q(t), \dq(t)) \in \mathcal{F}           & \forall t \ge 0 \\
& |\ddq(t)| \le \ddq^{max}        & \forall t \ge 0 \\
& q(0) = q_0, \quad \dq(0) = \dq_0  &
\end{aligned} \end{equation}
The solution of this problem is rather intuitive: the maximum initial velocity is such that, if we constantly apply the maximum deceleration, we end up exactly at $q^{max}$ with zero velocity.
Dealing with this problem implies taking into account the disturbance $w$. To solve it we can:
\begin{enumerate}
    \item write the position trajectory for constant acceleration and disturbance \mbox{$\ddq(t) =-\ddq^{max}+w$}, 
    \item compute the time at which the velocity of this trajectory is zero \mbox{$t^{0} = \dq_0 /( \ddq^{max}+w)$}, 
    \item compute the initial velocity such that $q(t^{0}) = q^{max}$
\end{enumerate}
So following the steps we find:
\begin{equation} \begin{aligned} 
    \dq^{\mathcal{V}}_M(q) = \sqrt{2 (\ddq^{max}+w) (q^{max} - q_0)}
    \end{aligned} 
\end{equation}
Knowing that $w$ is bounded we can rewrite this expression in order to deal with the worst-case scenario: a positive disturbance that reduce the deceleration and lead as a consequence to a smaller value of $\dq^{\mathcal{V}}_M(q)$
\begin{equation} \begin{aligned} 
    \dq^{\mathcal{V}}_M(q) = \sqrt{2 (\ddq^{max}-\bar{w}) (q^{max} - q_0)}
    \end{aligned} 
\end{equation}
Following the same steps we can define also the minimun velocity to ensure viability
% such that, by constantly applying the maximum acceleration, we end up exactly at the lower position limit $q^{min}$ with zero velocity, which leads us to:
\begin{equation} \begin{aligned} 
    \dq^{\mathcal{V}}_m(q) = - \sqrt{2 (\ddq^{max}-\bar{w}) (q_0 - q^{min})}
    \end{aligned} 
\end{equation}
So the set $\mathcal{V}^C$ of the feasible state can be rewritten as
\begin{equation} \begin{aligned} 
    \mathcal{V}^c = \{ (q,\dot{q}) : (q, \dq) \in \mathcal{F}, \, \dq^{\mathcal{V}}_m(q) \le \dq \le \dq^{\mathcal{V}}_M(q) \}
    \end{aligned} 
\end{equation}
This definition of $\mathcal{V}^c$ allows us to check easily the viability of a state by just verifying three inequalities.

[Image of State Space with different error => how the Viability change]

\subsubsection*{Discrete-time control}
The main difference between the continuous-time control and discrete-time is the fact that in the first one we can change the value of the acceleration at any instant and the other one we must keep the same value of acceleration for the whole timestep. This means that if we reach the $q^{max}$ position with zero velocity with a continuous control we can switch to zero also the $\ddq$ value, instead if reach the $q^{max}$ position with zero velocity with a discrete control we must apply a constant acceleration $\ddq=-\ddq^{max}$ for the whole timestep. Theoretically this can lead to a violation to the velocity bound or the lower position bound. Let us understand in which condition this behaviour is possible.
The worst case is represented by the system reaching the state $(q^{max},0)$ a moment after the beginning of the time step with the maximum deceleration and the disturbances $w=-\bar{w}$ that increase the deceleration effect. The resulting deceleration must be kept for all the time step. 
\begin{center}
    \begin{cases} 
        q_1 = q^{max}-0.5\dt^2(\ddq^{max}+\bar{w}) \\ \dq_1 = -\dt(\ddq^{max}+\bar{w})
    \end{cases}
\end{center} 
First of all it is necessary that the value of $\dq_1$ does not violate the velocity bound in one time step so $\dq^{max}\geq\dt(\ddq^{max}+\bar{w})$. After this time step the robot is approaching the lower bound with $\dq_1$. To ensure that the lower bound is not going to be reached it is necessary to obtain an inversion of the sign of $\dq_1$ . Maximum acceleration must now be applied, and $-\bar{w}$ still represent the worst-case error, decreasing the value of acceleration.
\begin{center}
    \begin{cases} 
        q_2 = q_1+\dq_1\dt+0.5\dt^2(\ddq^{max}-\bar{w}) \\ 
        \dq_2 = \dq_1+ \dt(\ddq^{max}-\bar{w})
    \end{cases}
\end{center} 
After the substitution of $q_1,\dq_1$:
\begin{center}
    \begin{cases} 
        q_2 = q^{max}-0.5\dt^2(\ddq^{max}+2\bar{w}) \\
        \dq_2 =-2 \dt\bar{w}
    \end{cases}
\end{center} 
The velocity, due to $-\bar{w}$ disturbance, has still a negative value. We are going to impose that with the next step the velocity is going to be positive. This is not strictly necessary, but ... . So we keep for another timestep the maximum acceleration and again $-\bar{w}$ disturbance and analyze in first place the velocity equation
\begin{equation}
    \dq_3 = -2 \dt\bar{w}+ \dt(q)(\ddq^{max}-\bar{w}) \Rightarrow \dq_3 =  \dt(q)(\ddq^{max}-3\bar{w}) 
\end{equation}
\begin{equation}
     \dt(q)(\ddq^{max}-3\bar{w}) \geq 0 \Rightarrow \bar{w}\leq \frac{1}{3}  \ddq^{max}
     \label{eq:limited_alpha}
\end{equation}
The assumption of a positive $\dq$ at the third step leads to accept as valid for our computation only disturbances that are one third of the maximum acceleration value. Is possible to deal with greater value of $\bar{w}$ relaxing this hypothesis or imposing this condition after an higher number of timestep, but our thoughts are that this $\bar{w}$ value is reasonable and is not useful add unnecessary complexity to the problem. 
After accepting this assumption we can say that in the third timestep the sign of $\dq$ is going to change for sure. Analysing the $\dq$ law in continuous time in third step make us able to calculate the time when the velocity goes to 0 and changes sign i.e. the time that the system reach the minimum value in its trajectory.
\begin{equation} \begin{aligned}
    \dq_3(t) = \dq_2+ t_0(\ddq^{max}-\bar{w}) =0 \\
    t_0 = \frac{-\dq_2}{\ddq^{max}-\bar{w}} \Rightarrow t_0=\frac{2 \dt \bar{w}}{\ddq^{max}-\bar{w}}
    \label{eq:t0_third_step}
    \end{aligned}
\end{equation}
We can now substitute $t_0$ in the position equation and obtain the value of the minimum point of the trajectory. This value is defined as $q^{discr}$ and represent in this case the maximum value of the lower bound that we can deal with using a discrete-time control.

\begin{equation}
    q_3(t_0) \triangleq q_{discr} = q_2 - \frac{4\dt\bar{w}}{\ddq^{max}-\bar{w}}\dt\bar{w} + \frac{\dt^2\bar{w}^2}{\ddq^{max}-\bar{w}}(\ddq^{max}-\bar{w}) 
\end{equation}
 The $q^{min}$ value of your system must be lesser or equal to $q^{discr}$ to not lead to a violation with those hypothesis.
\begin{equation}
    q_{discr} \triangleq q^{max}-\dt^2(\ddq^{max}+2\bar{w}+\frac{\bar{w}^2}{\ddq^{max}-\bar{w}}) \ge q^{min}
\end{equation}

 \begin{equation}
     q^{max}-q^{min}\le\dt^2(\ddq^{max}+2\bar{w}+\frac{\bar{w}^2}{\ddq^{max}-\bar{w}})
     \label{eq:final_cons}
 \end{equation}
Looking at \ref{eq:final_cons} is possible to notice that when the disturbance $w=0$ the expression is exactly the one found in \cite{DelPrete2018}, as expected. It is also possible to notice that theoretically the effective acceleration of the system $(\ddq^{max}+2\bar{w}+\frac{\bar{w}^2}{\ddq^{max}-\bar{w}})$ do not care in the re partition between the contribution of $\ddq^{max}$ or $w$, in order to respect the bound we have only to ensure $\frac{q^{max}-q^{min}}{\dt^2}\le(\ddq^{max}+2\bar{w}+\frac{\bar{w}^2}{\ddq^{max}-\bar{w}})$. Now our concern is to find the maximum allowable value of $w$ that keep that expression verified. It is possible to rewrite in the form
\begin{equation}
    aw^2+bw+c\leq0
    \label{eq:gen_2_ord}
\end{equation}
where 
\begin{itemize}
    \item $a = \dt^2$
    \item $b = q^{min} - q^{max}- \dt^2\ddq^{max}$
    \item $c = (q^{max}-q^{min}\ddq^{max}-\dt^2\ddq^{max})$
\end{itemize}
Is reasonable to rewrite the disturbance $w$ as a function of $\ddq^{max}$, in particular we can rewrite the ultimate bound value \bar{w} as
\begin{equation}
    \bar{w}= \alpha\ddq^{max}
\end{equation}
where $\alpha$ is a scalar. The final expression of \ref{eq:gen_2_ord} is
\begin{equation}
    \dt^2\ddq^{max}^2\alpha^2 + (q^{min}-q^{max}-\dt^2\ddq^{max})\ddq^{max}\alpha - \dt^2\ddq^{max}^2 + \ddq^{max}q^{max} - q^{min}\ddq^{max}\leq0
    \label{eq:final_alpha_eq}
\end{equation}
The solution of \ref{eq:final_alpha_eq} is an allowable range of $\alpha$ delimited by the two extreme value $\bar{\alpha}$
\begin{equation}
     \bar{\alpha}=\frac{-\alpha \ddq^{max}b \pm \sqrt{(\alpha \ddq^{max} b)^2-4c}}{2\alpha \ddq^{max} a}
     \label{eq:alpha_inequality_limit}
\end{equation}
% We have to discard the positive one but still i cannot say why
\subsubsection*{Numerical example}
It's possible to understand the behaviour of this equation and the previously stated limit on the disturbance set in \ref{eq:limited_alpha} showing a numerical example. We took in consideration a joint whose positions limits and controlling $\dt$ are
\begin{equation} \begin{aligned}
    q^{min}=-0.1  \ \ q^{max}=0.1 \ \ \dt=0.1
    \end{aligned}
\end{equation}
Looking at \ref{eq:final_cons} is possible to set the maximum effective acceleration that the system can take into account as 
\begin{equation}
    q^{allow}=\frac{q^{max}-q^{min}}{\dt^2} \Rightarrow q^{allow} = 20
\end{equation}
and it is possible to make a plot \ref{fig:Allowable_alpha} showing the allowable disturbance that the system can deal with as a function of the joint acceleration. It is possible to notice that with lower value of $\ddq^{max}$ the maximum $\alpha$ value derive from the imposed condition in \ref{eq:limited_alpha}. Crossing the point where \ref{eq:alpha_inequality_limit} is equal to $\alpha=\frac{1}{3}$, \ref{eq:alpha_inequality_limit} become a more strict bound.

\begin{figure}[H]
    \centering
    \includegraphics[scale=0.4]{aalpha.eps}
    \caption{Maximum allowable $\alpha$ value varying $\ddq^{max}$ value  }
    \label{fig:Allowable_alpha}
\end{figure}
As we can notice from this numerical example, the normal working condition of a robot are quite far from the constraints derived here. Assuming a large $\dt$ as $0.1$ require ... Since the working with this condition is of limited practical utility, we are going to assume that the \ref{eq:final_alpha_eq} is verified and so the viability kernel in the discrete time coincide with the viability kernel in discrete time $\mathcal{V}=\mathcal{V}^{C}$

\subsection{Reformulate in term of Viability}
We have now obtained a formulation of $\mathcal{V}$ so we can reformulate the problem \ref{eq:viab_ineq2}. Starting from the current state $(q,\dq) \in \mathcal{V}$ we need to compute the maximum value of $\ddq$ such that:
\begin{enumerate}
    \item the next state $(q(\dt,\dq(\dt)))\in \mathcal{V}$
    \item the entire trajectory leading to the next state belongs to the feasible set $\mathcal{F}$.
\end{enumerate}
The first condition alone is not sufficient, similarly to what happens in \ref{eq:t0_third_step} ,because the trajectory between two viable state may violate a constraint. Viability only ensure the existence of future feasible trajectory, so starting from a viable state we can violate a constraint. We can reformulate the \ref{eq:viab_ineq2} as:
\begin{equation} \label{eq:problem_ref_viab} \begin{aligned} 
    \ddq_0^{max} = \maximize_{\ddq} \quad & \ddq \\
    \st \quad & q(0) = q, \quad \dq(0) = \dq \\
    & (q(t), \dq(t)) \in \mathcal{F}	\qquad	 \forall t \in [0, \dt] \\
    & \dq^{\mathcal{V}}_m(q(\dt)) \le \dot{q}(\dt) \le \dq^{\mathcal{V}}_M(q(\dt)) \\
    & |\ddq| \le \ddq^{max}
    \label{eq:reformulate_viability}
    \end{aligned} 
\end{equation}
We can notice that the problem is much simpler than the previous one: it has a single variable instead of a infinite sequence and its constraints concern only the trajectory in the $[0,\dt]$ interval,rather then in $[0,\inf]$. As stated in \cite{DelPrete2018} the problem \ref{eq:reformulate_viability} is still hard to solve because the constraints are infinitely many and nonlinear, but we can rewrite the inequality constraints in order to obtain an upper bound and lower bound for our $\ddq$
\subsubsection*{Position Inequalities}
The position trajectory with disturbance is 
\begin{equation} \begin{aligned}
    q^{min} \leq q+t\dq+ \frac{1}{2}t^2(\ddq+w)\leqq^{max} \ \forall t\in[0,\dt]
    \label{eq:pos_traj_ineq}
    \end{aligned}
\end{equation}
We want this inequality to be respected and remain bounded between its limits for the whole timestep and not only at the end with $t=\dt$. If the velocity $\dq$ does not change its sign during the timestep its actually enough to verify the condition only in $t=\dt$. However if the sign change in the middle of timestep the position inequality can be still satisfied in $t=0$ and in $t=\dt$, but leads to a violation somewhere in the middle. Lets us focus first on the upper bound following the \cite{DelPrete2018} approach: we can rewrite the upper-bound of constraint in \ref{eq:pos_traj_ineq} as:
\begin{equation}
    f(\ddq)\leq q^{max}
    \label{eq:func_costraint}
\end{equation}
where:
\begin{equation} \begin{aligned} 
    f(\ddot{q}) &= \maximize_{t \in [0, \dt]} \, [ q + t \dot{q} + 0.5 t^2 (\ddot{q} +w)]
    \end{aligned} 
\end{equation}
Following a procedure similar to \ref{eq:t0_third_step} we want to find the time to which the position reach its maximum so we set the time derivative of the position to zero and retrieve the correspondent value:
\begin{equation}
    \begin{aligned}
    \dq+t(\ddq+w)=0 \rightarrow t^{ext} \triangleq -\dq/(\ddq+w)
    \end{aligned}
\end{equation}
The extremum is a maximum and its reached during the timestep only if:
\begin{equation}
    \begin{aligned}
    \dq\ge0, \ \ddq\leq-\frac{\dq}{\dt}-w \triangleq \ddq_1^{M}
    \end{aligned}
    \label{eq:position_timestep_condition}
\end{equation}
If these conditions are satisfied the maximum position is: 
\begin{equation}
    \begin{aligned}
    f(\ddq) = q+t^{ext}\dq+\frac{1}{2}(t^{ext})^2(\ddq+w) \\
    f(\ddq) = q-\frac{\dq^2}{2(\ddq+w)}
    \end{aligned}
\end{equation}
and so if we substitute in the \ref{eq:func_costraint} it becomes \ref{eq:same_as_minus_w_position}: it is possible to notice that it is equivalent to found the upper-bound without disturbance as in \cite{DelPrete2018} and subtract the $w$ disturbance.  
\begin{equation}
    \begin{aligned}
    \ddq \leq \frac{\dq^2-2qw+2q^{max}}{2(q-q^{max})} \\
    \ddq \leq \frac{\dq^2}{2(q-q^{max})}-w \triangleq \ddq^M_2
    \end{aligned}
    \label{eq:same_as_minus_w_position}
\end{equation}
If the conditions in \ref{eq:position_timestep_condition} are not satisfied the maximum point of position trajectory is reached at the boundary of the timestep and we must only ensure that the constraint is verified for $t=\dt$:
\begin{equation}
    \ddq \leq \frac{2}{\dt^2}(q^{max}-q-\dt \dq) -w \triangleq \ddq^M_3
\end{equation}
[to finish]
\subsubsection*{Velocity Inequalities}
Velocity trajectory is a line and so we need to consider only the extremes of this trajectory, so verify that the bound is satisfied for $t=\dt$:
\begin{equation} \label{eq:vel_ineq} \begin{aligned} 
\frac{1}{\dt} (- \dot{q}^{max} - \dot{q})+w \le \ddot{q} &\le \frac{1}{\dt} (\dot{q}^{max} - \dot{q})-w
\end{aligned} \end{equation}
\subsubsection*{Viability inequalities}
Let us consider the upper bound of the viability inequality:
\begin{equation} \begin{aligned} 
    \dq(\dt) \leq \sqrt{2 (\ddq^{max}-\bar{w}) (q^{max} - q(\dt))} \\
   \dq+\dt\ddq \leq \sqrt{2 (\ddq^{max}-\bar{w}) (q^{max} - q-\dt\dq-0.5\dt^2\ddq)}
    \end{aligned} 
\end{equation} 

\end{document}